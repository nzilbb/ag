\documentclass{article}
\usepackage[utf8]{inputenc}

\pagestyle{plain}
\setlength{\parindent}{0pt}
\setlength{\parskip}{1ex plus 0.5ex minus 0.2ex}
% command for speaker turns...
% copy this into your preamble if you copy parts of the transcript into another document:
\newcommand{\turn}[2]{
\item[#1:] #2
}
\begin{document}
\begin{center}\textbf{webvtt}\end{center}

\begin{description}

\turn{speaker}{Before we move to the first question to the First Minister, I invite the First Minister\\
to make a few remarks following the tragic events in Christchurch in New Zealand.\\
The First Minister (Nicola Sturgeon): Thank you, Presiding Officer. I begin today\\
with heartfelt condolences to the people of New Zealand after last week’s appalling\\
terrorist attack in Christchurch. I hope that people in New Zealand can take some comfort\\
from the knowledge that people across the world stand in solidarity with them.\\
Events in New Zealand have been felt deeply here in Scotland, as in other countries, and\\
perhaps especially in our Muslim community. Last week, Police Scotland arranged reassurance\\
patrols and visits to mosques and other places of worship. On Friday, I visited Glasgow central\\
mosque with the justice secretary.\\
The Prime Minister of New Zealand has said that nations around the world are engaged\\
in a global fight against far-right, racist and extremist ideology. Regrettably, she is\\
absolutely right. All of us have a responsibility to engage in that fight. We must tackle hatred\\
and prejudice through the words that we use, the actions that we take and the climate that\\
we create. I know that all parties in this chamber will play their part in doing that.\\
In the past week, we have also seen an attack in Utrecht and the stabbing of a teenager\\
in Surrey. Our condolences are with all those who have been affected by those incidents\\
as well.\\
Let us today express sympathy and solidarity with the victims of racist and extremist violence\\
in Christchurch and around the world. Above all, let us make clear our determination that\\
the proponents of hate will be defeated by the values of kindness, compassion and love.\\
[Applause.]\\
The Presiding Officer: We turn to questions to the First Minister,\\
the first of which is from Jackson Carlaw.\\
Jackson Carlaw (Eastwood) (Con): I associate all of us in the Scottish Conservatives\\
with the First Minister’s remarks and offer our support for any measures that are required\\
to reassure those who attend mosques in Scotland.\\
Many Scots will have friends and family who live in or regularly visit New Zealand and\\
who will have been deeply affected. However, for many in Scotland’s Muslim community,\\
events on the other side of the world must never have felt closer to home. As we embrace\\
them with our good wishes and condolences, we must—as the First Minister said—work\\
together to think afresh about what must be done by us all to counter this defining 21st\\
century scourge.\\
Over the past 10 years, the Scottish National Party Government has launched two major drug\\
strategies. Tragically, during those 10 years drug deaths have doubled. We are now on course\\
to have the largest number of drug deaths per head anywhere in Europe. Does the First\\
Minister believe that the strategies have been a success or a failure?\\
The First Minister (Nicola Sturgeon): These are challenging issues and I readily\\
concede that this Government—any Government—must remain open to fresh thinking and new ideas.\\
The situation with regard to drug deaths is not one that any of us would consider to be\\
acceptable. However, as I said in the chamber last week, many of those who have died have\\
lived with alcohol and drug use for a long time—such people become more vulnerable\\
as they grow older as a result of their complex health and social needs. Although I do not\\
overstate the point, more encouraging is the fact that the last report showed fewer deaths\\
among the under-25 population. Recent reports also highlight falling heroin use—again,\\
particularly among under-25s.\\
As I am sure that Jackson Carlaw is aware, work is under way in Dundee and Glasgow to\\
consider what more can be done to tackle drug deaths. That work will be of relevance around\\
Scotland, but we want to see the outcomes from it before we consider what further action\\
we should take.\\
Jackson Carlaw: We all want to sort the crisis, but the first\\
step is surely to admit that the current policy is not working as it should. Regrettably,\\
it has been a failure. I have an example of that.\\
We know that rehabilitation services in prisons can be vital in turning around people’s\\
lives. However, my colleague Adam Tomkins has discovered in recent days that in Barlinnie,\\
which is one of our biggest prisons, a successful voluntary project—a recovery cafe where\\
people can go to get their lives back on track—is facing closure. How can it be right that we\\
prioritise spending millions of pounds on methadone programmes, yet successful projects\\
such as the cafe are put at risk?\\
The First Minister: First, I say to Jackson Carlaw and to Adam\\
Tomkins that the justice secretary has received a letter on the issue, which will be responded\\
to in due course.\\
It is important that I advise members that the Scottish Government has not previously\\
funded recovery cafes. However, we provide funding for the Scottish Recovery Consortium,\\
and the Scottish Prison Service adopts a therapeutic approach in dealing with addiction issues\\
and provides support for those with addiction problems who are in their care.\\
The new alcohol and drug strategy highlights the importance of recovery communities and\\
the need for them to be at the heart of any proposals. They help to reduce stigma, because\\
they provide the visible face of recovery, as well as insights into addiction and harm.\\
Through our sustained funding of the Scottish Recovery Consortium, we will continue to do\\
what we can to support the growth of recovery communities across the country.\\
We will of course give consideration to the points that Adam Tomkins made in his letter.\\
Jackson Carlaw: My question was not intended as a criticism\\
of that mix of approaches.\\
Just a few miles from Parliament is Castle Craig hospital near West Linton, which is\\
a drug rehabilitation centre with capacity for residential drug rehab patients, and which\\
the Conservative health spokesman, Miles Briggs, visited recently. Hospital staff told him\\
that Castle Craig is not receiving national health service referrals and is mostly kept\\
going by patients who are referred from the Netherlands for treatment. Is not the First\\
Minister, like me, concerned that Dutch patients are getting better access to that rehabilitation\\
project here in Scotland than local Scots who are in need of the same support and treatment?\\
The First Minister: I am very happy to look into that specific\\
example. We want people to have access to a broad range of rehabilitation services.\\
The Scottish Government is providing £70 million this financial year to reduce the\\
harms that are caused by alcohol and drugs. That includes an additional £20 million for\\
drug and alcohol services, which is being allocated to support new approaches, so that\\
we respond in a much more joined-up and person-centred way. Such investment is important.\\
I am not trying to make a party-political point about a very serious issue, but it is\\
also important that we are prepared to take forward innovative and evidence-based new\\
approaches, even if at first they seem to be challenging, particularly for public opinion.\\
That is why we supported the principles behind Glasgow’s proposals for a medically supervised\\
safer drug consumption facility and heroin-assisted treatment in the city. It is important that\\
we work with health and social care partnerships on new approaches, as well as ensuring that\\
we invest in rehabilitation. I hope that the Conservatives will think about giving us support\\
on that, because we need to persuade the United Kingdom Government to do what is required.\\
Jackson Carlaw: I respect the First Minister’s approach\\
to that policy. We have looked at it, but, unfortunately, it is the one policy in this\\
area on which we fundamentally disagree. We think that the policy should be to get people\\
clean of drugs, not to provide opportunities for people to take them.\\
Scottish Conservatives have set out a clear plan to tackle Scotland’s growing drugs\\
crisis, which is to get first-time offenders into treatment, direct more money into rehabilitation\\
programmes run by third sector bodies and, at the same time, at least review the failed\\
methadone programme.\\
Let us admit that, in politics at the moment, we are not overrun with issues on which we\\
can form consensus. However, on this one vital issue, will the First Minister commit today\\
to working across the chamber—we will commit to that—to improve the drugs strategy for\\
the next 10 years, so that we can cut drug deaths and drug addiction and come down hard\\
on those peddling misery in our communities?\\
The First Minister: I reiterate my willingness to work across\\
the chamber. I think that I have said in a couple of my responses today that I will consider\\
the points that Jackson Carlaw has raised, and I give that reassurance again. I ask for\\
the same in return. I am slightly concerned at the almost knee-jerk way in which Jackson\\
Carlaw ruled out the fresh thinking around safer drug consumption facilities. If we are\\
genuinely to try and find a consensus, we have to be open to new thinking, and that\\
will sometimes be very tough and challenging. I appeal to Jackson Carlaw to reconsider his\\
opposition to that policy, just as he is asking me to be open-minded to any proposals that\\
he makes.\\
We will continue to ensure that we have the right strategies in place to deal with what\\
we all accept is a challenging and complex issue. First, that involves taking a very\\
hard line against those who supply drugs—and we saw figures earlier this week about police\\
seizures of drugs. Secondly, it definitely involves support, particularly rehabilitation\\
support, for those who are addicted to drugs. Thirdly, it involves being open to new ideas\\
and new thinking. If we can all agree broadly around that approach, perhaps we can build\\
a consensus that allows us to tackle something that we all agree is unacceptable. We want\\
to see a considerably improved situation, and I hope that we have the support of Jackson\\
Carlaw and the Conservatives on that.\\
Richard Leonard (Central Scotland) (Lab): I add the deep-felt condolences of the Scottish\\
Labour Party to the families and friends of all those who lost their lives in the terror\\
attack in Christchurch last Friday. I offer our support for practical action to defeat\\
racism and hatred wherever it occurs.\\
To ask the First Minister why there is a staffing crisis in the national health service.\\
The First Minister (Nicola Sturgeon): There is not a staffing crisis in the national\\
health service. There are record numbers of people working in the national health service.\\
In fact, I can tell Richard Leonard that staffing levels in NHS Scotland are now at a record\\
high and are up by more than 13,600 since 2006, just before this Government took office.\\
The number of consultants is up by 51 per cent; the number of qualified nurses and midwives\\
is up by 8 per cent; and there is a higher level of NHS staffing per head in Scotland\\
than there is in NHS England.\\
Our NHS staff of course work under considerable pressure, and we are grateful to them for\\
the job that they do, but we will continue to invest in our NHS to ensure that there\\
are record numbers of staff, so that they can continue to deliver the excellent services\\
that they do.\\
Richard Leonard: This week, the Parliament’s Health and Sport\\
Committee, following the tragic events at the Queen Elizabeth university hospital, began\\
its inquiry into infection control standards. New figures released to Scottish Labour this\\
week reveal that the number of domestic staff—that is, cleaners—who are employed at the Queen\\
Elizabeth university hospital is falling. In March 2018, 464 cleaners were employed\\
at the hospital. According to the latest figures, that number has dropped to 440. Why, at the\\
very point when it is facing a rise in infection outbreaks, is Scotland’s biggest hospital\\
employing fewer people on the front line whose job it is to keep that hospital clean and\\
safe?\\
The First Minister: I am sure that Richard Leonard will have heard\\
the Cabinet Secretary for Health and Sport already address this issue publicly. The issue\\
has been raised with Greater Glasgow and Clyde NHS Board. It is absolutely imperative that\\
all health boards in all hospitals ensure appropriate numbers of domestic and cleaning\\
staff.\\
It is of course for health boards to consider the configuration of staffing. As Richard\\
Leonard will know, and as those of us who represent Glasgow constituencies know particularly\\
well, there has been a significant change in the configuration of Glasgow hospitals\\
over the past number of years, and the overall staffing numbers will undoubtedly reflect\\
that.\\
We will continue to raise issues directly with health boards to ensure that they are\\
addressed where that is necessary. Notwithstanding the very serious incidents at the Queen Elizabeth\\
university hospital, which we have discussed on many occasions in the chamber before—and\\
I welcome the Health and Sport Committee’s inquiry into these issues—infection rates\\
are down considerably in Scottish hospitals overall.\\
I see that Jackie Baillie is in the chamber. She and I regularly used to have exchanges\\
about the levels of Clostridium difficile in our hospitals, following the tragic incident\\
at the Vale of Leven hospital. C diff, MRSA and infections generally are down, in some\\
cases by more than 80 per cent.\\
Let us tackle issues where they arise—Richard Leonard is right to raise them—but let us\\
not lose sight of the good work that has been done in our NHS to reduce infection and to\\
put a real focus on patient safety.\\
Richard Leonard: I should also make it clear that the problem\\
is not unique to one hospital: it is replicated right across the NHS Greater Glasgow and Clyde\\
area. There are fewer domestics, porters and laundry and linen staff compared with last\\
year’s levels. It is clear that we have a staffing crisis in our health service, and\\
that it is not confined to consultants, nurses and midwives but extends to facilities staff,\\
domestics, catering workers, porters and laundry staff—all workers without whom no hospital\\
can operate.\\
We know that there is a parliamentary inquiry and that reviews are being carried out by\\
the health board and the Government. However, these issues are serious and urgent. The public,\\
and the staff who are under pressure, need to hear a commitment that the reduction in\\
such vital front-line jobs will be reversed as soon as possible. Is the First Minister\\
prepared to give them that commitment today?\\
The First Minister: As I said earlier, we will continue to work\\
with health boards, including NHS Greater Glasgow and Clyde, to ensure that they have\\
appropriate staffing levels across all specialties in the NHS. That is important. I repeat what\\
I have already said: record numbers of staff are working in our national health service.\\
Richard Leonard says that the issues are urgent, and I could not agree more. I know how devastating\\
outbreaks of infection in hospitals are—principally for patients and their families, but also\\
for the staff who work there. That is why the Healthcare Environment Inspectorate’s\\
report on the Queen Elizabeth university hospital, which was commissioned and instructed by the\\
Cabinet Secretary for Health and Sport, has already been completed, and why its recommendations\\
have already been accepted by the health board and are being implemented.\\
Whatever disagreements we might have, and whatever legitimate points Richard Leonard\\
might raise—they are legitimate points—I do not think that anybody could doubt the\\
seriousness of the Government and the health service when it comes to tackling infections\\
in our hospitals. Overall, the figures state that things are going in the right direction,\\
but that does not take away from the need to tackle serious incidents when they arise.\\
We will continue to do exactly that.\\
The Presiding Officer: We turn to constituency supplementary questions,\\
the first of which is from Liam Kerr.\\
Liam Kerr (North East Scotland) (Con): Cove harbour fishing community is suffering.\\
First, its landing was bought—and closed—by a private landlord. It went to court and won\\
rights of public access but faced significant legal costs. Several boats were then destroyed\\
in a fire, and now the landlord has closed access to the beach. Community representatives\\
have written to the Cabinet Secretary for the Rural Economy, Fergus Ewing, several times,\\
requesting a meeting—even if that were to be here at Holyrood—to discuss their rights\\
and their future, but to no avail.\\
Will the First Minister ask the cabinet secretary to meet those representatives, and not risk\\
ignoring a community that faces the loss of its livelihood?\\
The First Minister (Nicola Sturgeon): Of course, the Scottish Government wants to\\
do everything possible to help any community that is experiencing difficulties. Beyond\\
what the member has just said, I am not aware of the content of the correspondence with\\
Fergus Ewing, but I am happy to give an undertaking to look into that and, if the cabinet secretary\\
thinks that the Scottish Government can offer help, for him to meet those who are affected.\\
Dr Alasdair Allan (Na h-Eileanan an Iar) (SNP): Earlier this week, the Office of Gas and Electricity\\
Markets announced that it was minded to reject proposals for a 600MW transmission link to\\
the Western Isles, saying that it would instead support a much-reduced 450MW link. That has\\
been met with extreme disappointment in my constituency, because it will severely constrain\\
capacity for future community projects and place other existing projects from the Western\\
Isles at a potential disadvantage.\\
What pressure can the Scottish Government put on Ofgem, and the United Kingdom Government,\\
to reconsider that short-sighted decision?\\
The First Minister (Nicola Sturgeon): The Scottish Government is absolutely committed\\
to unlocking the vast renewables potential of our islands and the associated economic\\
benefits for our island communities. We are very concerned at the uncertainty over the\\
proposed connection from the Western Isles. The Government believes that for the islands’\\
full renewables potential to be realised, a larger link is required, so I very much\\
agree with the sentiment of Alasdair Allan’s question. We have made arguments directly\\
to Ofgem to support that point, and we will continue to do so as we engage further with\\
it and with island stakeholders and developers during the on-going consultation process.\\
I assure Alasdair Allan—and the chamber—that we will make absolutely every effort to secure\\
the right outcome for the Western Isles.\\
Jackie Baillie (Dumbarton) (Lab): The First Minister will be aware that my constituent\\
Jagtar Singh Johal has spent more than 500 days detained in prison in the Punjab. There\\
have been accusations of torture and he has now faced his 77th pre-trial preliminary hearing.\\
His MP, Martin Docherty-Hughes, is to be commended for pursuing the matter vigorously.\\
Will the First Minister use her influence and speak to the Foreign Secretary and the\\
United Kingdom Government to urge them to provide support and assistance to Mr Johal\\
and his family?\\
The First Minister (Nicola Sturgeon): I thank Jackie Baillie for raising this issue.\\
I know that she has raised it previously and she is right to say that Martin Docherty-Hughes\\
MP has been assiduous in raising the rights and situation of his constituent.\\
We have raised this issue and we will continue to do so. The Deputy First Minister has raised\\
it directly with Indian ministers on recent visits to India and with the British high\\
commission. I believe—although I will double-check this—that we have raised the issue directly\\
with the Foreign Office. If not, I am happy to undertake that we will do so.\\
Miles Briggs (Lothian) (Con): In recent weeks, I have received correspondence\\
from families across Scotland who are facing unacceptable waits for cleft surgery. Two\\
years ago, we warned Scottish National Party ministers about the impact of the closure\\
of the Edinburgh unit and the centralisation of cleft services. This Parliament voted against\\
centralisation, but ministers pressed on against the will of Parliament.\\
One case highlighted to me just this week is that of a young man who has been waiting\\
two years for a promised final surgery and is no further forward on when he will receive\\
that. Families are also telling me that they are looking to NHS England in order to receive\\
the surgery. Will the First Minister apologise to families for those waits? What will she\\
do to correct the mistake that this Government made?\\
The First Minister (Nicola Sturgeon): As I have said many times before in the chamber,\\
I regret it when any patient has to wait longer for treatment than we would want to be the\\
case.\\
On the issue of cleft surgery, as I recall, the redesign of that service was on clinical\\
grounds, to ensure a quality and safe service. If Miles Briggs would like to give further\\
details of the constituents who are raising issues with him, the Cabinet Secretary for\\
Health and Sport will look into those and, once she has had the opportunity to do that,\\
will correspond further with him.\\
Patrick Harvie (Glasgow) (Green): On behalf of the Scottish Greens, I join others\\
in expressing our shared concern for the bereaved and injured following the far-right terrorist\\
attack in New Zealand, but also our respect for the response that that country is showing,\\
recommitting to the values of its inclusive society and refusing to placate the far right,\\
as far too many politicians around the world have done.\\
Last night, in the midst of a crisis of her own making, the Prime Minister again refused\\
to listen to reason and instead effectively told the public that Parliament is their enemy.\\
Scotland needs the freedom to take a different direction, leave behind this chaos and find\\
our own way out of the crisis. That is why we need our independence. The First Minister\\
told us that she would say something about her preferred timing within weeks. That was\\
two months ago. I ask again, when?\\
The First Minister (Nicola Sturgeon): First, I agree with Patrick Harvie that the\\
Prime Minister’s comments last night were deeply irresponsible and I hope that, in time,\\
she will reflect on that.\\
The Prime Minister’s comments also failed to accept any of the responsibility that she\\
bears for the mess that the United Kingdom is in right now. She wanted to blame everybody\\
except herself, and yet I think that most people know that it was the Prime Minister\\
who triggered article 50 without a plan. It was the Prime Minister who drew self-defeating,\\
contradictory red lines that boxed her in from the start. It was the Prime Minister\\
who called an unnecessary general election and who delayed the first vote on her deal\\
in an attempt to run down the clock. It was the Prime Minister who failed to listen and\\
change course after the first defeat of her deal and then again after the second. She\\
must change course now before it is too late and she must bear responsibility for the mess\\
that this country is in.\\
On the issue of independence, the frustration that people feel right now at Scotland’s\\
future being determined by the Democratic Unionist Party and a cabal of right-wing Tories\\
is understandable, and I absolutely share it. I said that I would wait until the end\\
of this phase of the Brexit negotiations before setting out my views on the way forward for\\
Scotland. Having done so this long, I think that it is reasonable for me to wait to see\\
what clarity emerges in the next few days, even if I suspect that it will just be clarity\\
that there will be no clarity. I will then set out my views on the path forward.\\
Nobody can be in any doubt that change is needed. The past three years have shown that\\
the status quo is broken. It cannot protect Scotland from the folly of Brexit and all\\
that flows from that. Even the most ardent unionist must see that the way we are now\\
governed by Westminster is broken. The question is how we fix that for the future, and there\\
is no doubt in my mind that letting people in Scotland choose an independent future is\\
the best way to do that.\\
Patrick Harvie: At every stage of this nightmare, this Parliament\\
has tried to persuade the Prime Minister to change course. We have called for the narrow\\
2016 result and Scotland’s remain vote to be respected, for our place in the single\\
market to be protected and for the public to have the right of a final say and the chance\\
to cancel this crisis. If the Prime Minister succeeds in closing off all those positive\\
choices and the country finds itself being driven to the edge of the cliff at this time\\
next week, does the First Minister agree that MPs must be prepared, finally, to put the\\
public interest first and willing, if all else fails, to do what is necessary and revoke\\
article 50?\\
The First Minister: Yes. Indeed, the Scottish National Party at\\
Westminster and the Greens, the Liberal Democrats and Plaid Cymru issued a joint statement last\\
night to that effect. SNP MPs will not vote for the Prime Minister’s deal, because it\\
is a bad deal that will damage Scottish interests. I do not think that any Scottish MP should\\
vote for such a deal. However, nor will we accept the Prime Minister framing it as a\\
choice between her deal and no deal. Just because she is not willing to contemplate\\
the alternatives does not mean that there are no alternatives. One of those alternatives\\
is, undoubtedly, revoking article 50. If all else fails by this time next week, that is\\
exactly what MPs should do.\\
Willie Rennie (North East Fife) (LD): I associate myself and my party with the First\\
Minister’s remarks about New Zealand. The events in that country were truly sickening.\\
You would not think that we were in the middle of a national crisis if you just listened\\
to the questions from the leaders of the Conservative and Labour parties, but the last thing this\\
country needs is more division and chaos with independence to compound the division and\\
chaos of Brexit. [Interruption.]\\
The Presiding Officer: Order, please.\\
Willie Rennie: The first duty of a Prime Minister is to keep\\
the country safe but, because of the cavalier choices of this Prime Minister, emergency\\
measures under operation yellowhammer have been triggered and medicines, food supply\\
chains and transport are all at risk. Does the First Minister agree that no serious Prime\\
Minister should ever threaten such catastrophic consequences, no matter how much she wants\\
her policy to be agreed?\\
The First Minister (Nicola Sturgeon): Before I address Willie Rennie’s question,\\
I say in response to the first part of what he said that the inconsistency in his position\\
is this: he wants people across the United Kingdom to have the ability to escape Brexit\\
through a second referendum—and I agree with him on that—but if that does not prove\\
to be possible, he thinks that Scotland should just grin and bear it, and put up with the\\
devastation of Brexit, instead of Scotland having the choice to escape Brexit and have\\
an independent future. That is a deeply inconsistent position for him to take and I hope that he\\
will reflect on it.\\
On operation yellowhammer, which is the emergency planning for a no-deal Brexit, it is beyond\\
comprehension that any Prime Minister could knowingly allow the country to be eight days—about\\
200 hours—away from the possibility of crashing out of the European Union without a deal and\\
to require that emergency planning work to be done. Yesterday, as I have done once a\\
week for several weeks, I chaired a meeting of the Scottish Government’s resilience\\
committee that was looking at medicine supplies, food supplies and transport links in the event\\
of a no-deal Brexit. It is outrageous that we have to expend time, energy and resources\\
on doing that. Before any more time passes and it is too late, the Prime Minister must\\
change course, take no deal off the table completely, look to build a broader consensus\\
rather than pandering to the hardliners in her own party and, if necessary, dump Brexit\\
completely. That would be in the best interests of the country.\\
Willie Rennie: The First Minister is wrong. The inconsistency\\
is to believe that breaking up an economic partnership of 40 years will be chaotic but\\
that breaking up one of 300 years will be a piece of cake. [Interruption.]\\
The Presiding Officer: Order, please.\\
Willie Rennie: The First Minister is the inconsistent one.\\
[Interruption.]\\
The Presiding Officer: Order, please.\\
Willie Rennie: People are scunnered by this agonising Brexit\\
process. We are three years on, with 200 hours left. Is it not time for a commonsense approach\\
under which the Prime Minister takes a no-deal Brexit off the table instead of using it as\\
a threat against her own citizens; all party leaders sit down and talk instead of the leader\\
of the Opposition walking out because he does not like Chuka Umunna; the Prime Minister\\
reaches out to MPs in Parliament rather than insulting them from behind a podium in number\\
10; and we admit that Parliament is incapable of deciding, so we have a public vote to let\\
the people decide? Is it not time for that commonsense approach?\\
The First Minister: Yes, I agree with all that. I think that people\\
across the UK should have the opportunity to vote again, given everything that they\\
now know that was not known in 2016. That is why I will be calling for that public vote\\
in London on Saturday, along with many others—no doubt, hundreds of thousands of others.\\
I agree with everything that the member said about the Prime Minister, and I share his\\
despair about the leader of the Labour Party and his childish behaviour last night at a\\
time when we need people to come together to find an alternative. Where I disagree with\\
the member is on his view that, if all of what he has just called for fails, Scotland\\
is powerless in the face of the disaster of Brexit. I oppose Brexit, as he does, but there\\
was nothing inevitable about the chaos of Brexit. That is down to those who proposed\\
it having no idea what it would look like in reality and doing no planning for it. It\\
did not have to be that way.\\
I say to Willie Rennie that the inconsistency is in him standing up to rightly spell out\\
what a disaster Brexit will be but then saying that, if all else fails, Scotland just has\\
to put up with it. I do not think that Scotland has to put up with it and I do not think that\\
Scotland should have to put up with it. If it comes to it, Scotland choosing independence\\
is a much brighter future than remaining part of Brexit Britain.\\
The Presiding Officer: We have some additional supplementaries.\\
Jenny Gilruth (Mid Fife and Glenrothes) (SNP): Last night, the Prime Minister claimed that\\
the public have had enough; today, a petition on the UK Parliament website calling for article\\
50 to be revoked is already well on the way to 1 million signatures. Support is growing\\
so fast that the website crashed harder than the Prime Minister’s credibility. If the\\
Prime Minister believes that the people are with her, should she not have the courage\\
to put that to the test and call for a people’s vote?\\
The First Minister (Nicola Sturgeon): Yes, I agree. As I said, I thought that the\\
Prime Minister’s statement last night was deeply regrettable. For her to blame everybody\\
except herself beggars belief. Now is the time for people across parties to speak out.\\
Last night, I watched one of the most powerful contributions that I have ever seen in the\\
House of Commons. It was from Dominic Grieve, a moderate Tory who I think everybody would\\
accept is an honourable person. He had the honesty to say that he was ashamed to be in\\
the Conservative Party and that the conduct of the Prime Minister made him want to weep,\\
yet Scottish Conservatives continue to parrot the lines of the Prime Minister. I often wonder\\
whether Jackson Carlaw ever, in his quieter moments, thinks that it might be better for\\
the country and indeed his own reputation for him to say what I believe he probably\\
thinks—that this is a mess, that carrying on regardless is a profound mistake and that\\
the Prime Minister must change course and must do so now before it is too late.\\
Daniel Johnson (Edinburgh Southern) (Lab): The First Minister will be aware that the\\
British Retail Consortium annual crime survey was published today. It records that, last\\
year, 115 shop workers were physically attacked at work every single day across the United\\
Kingdom. The Union of Shop, Distributive and Allied Workers—USDAW—estimates that the\\
real problem could be much greater; its estimate is that 34 retail workers are attacked every\\
day in Scotland alone.\\
My bill to protect shop workers is in the final stages of drafting. What does the First\\
Minister think needs to be done to tackle this growing problem and will her Government\\
work with me to look at what changes in the law may be needed to do so? Everyone has the\\
right to be safe at work, whether they work in an office or on a shop floor.\\
The First Minister (Nicola Sturgeon): I thank Daniel Johnson for raising the issue\\
and the results of the British Retail Consortium’s survey. It is a powerful reminder that our\\
shop workers do an essential job that is often dangerous to them, for which we all owe them\\
a huge debt of gratitude.\\
We will be happy to work with Daniel Johnson and others to look at what further protections\\
we need to put in place. He said that his bill is in the final stages of drafting; we\\
will look carefully at it when it is published and we will be happy to consider it and discuss\\
it with him. We will be happy to try to build consensus.\\
Mark McDonald (Aberdeen Donside) (Ind): On 1 April, employer contributions to national\\
health service pension schemes will increase from 14.9 to 20.9 per cent. Children’s hospices\\
across Scotland have estimated that the increased cost to them will be equivalent to the salaries\\
of nine full-time nurses. The United Kingdom Government has stated that funding for charities\\
and hospices is included in the funding that has been provided to NHS England to cover\\
the costs of the increase, but the Children’s Hospice Association Scotland says that similar\\
commitments have not yet been made to Scottish charities and hospices. Will the Scottish\\
Government provide funding to help charities and hospice organisations to meet the cost\\
increase and ensure that they do not have to divert money from the vital support services\\
that they provide?\\
The First Minister (Nicola Sturgeon): I thank Mark McDonald for rightly raising\\
the issue, which is concerning generally and in particular for hospices and charities.\\
The Scottish Government has been in discussions with the British Medical Association about\\
how best to disburse additional funding to practices to meet the change. We will continue\\
to discuss that, and I will ask the health secretary to look at the position of hospices\\
and charities and to come back to Mark McDonald when she has done so.\\
Christine Grahame (Midlothian South, Tweeddale and Lauderdale) (SNP):\\
To ask the First Minister, in light of the reported travel chaos on the Borders railway\\
last weekend as a result of a number of train cancellations, whether the Scottish Government\\
considers that the ScotRail franchise continues to be sustainable. (S5F-03189)\\
The First Minister (Nicola Sturgeon): I am disappointed that passengers across a\\
number of routes on the ScotRail network continue to be affected by train cancellations as a\\
consequence of ScotRail’s training backlog. There is evidence of some improvement in ScotRail’s\\
performance nationally, but that will do little to reassure passengers who attempted to travel\\
on the Borders railway last Sunday and were faced with an unacceptable number of cancellations.\\
That is why ScotRail’s focus must remain on delivering a robust remedial plan that\\
puts passenger interests at the forefront of restoring performance levels. The remedial\\
plan has been specifically designed to militate against train crew and train fleet challenges,\\
and I fully expect ScotRail to ensure that the plan is delivered in order to reaffirm\\
passenger confidence in the railway.\\
Christine Grahame: There was, indeed, a service meltdown. It\\
was a breakdown, too—on the Borders railway on Sunday. The cancellations continued since\\
then and continue today.\\
I heard the Cabinet Secretary for Transport, Infrastructure and Connectivity talking this\\
morning about remedial notices. The second notice that was served requires that a plan\\
be delivered soon. The plan might be delivered, but it will not deliver trains—plans do\\
not drive trains. Is not it time that the Scottish Government told Abellio ScotRail\\
that it is in the last chance saloon? I certainly think so, and so do my constituents.\\
The First Minister: ScotRail should treat the remedial plan very\\
much as the last chance saloon. That is the nature of it. ScotRail has been left in no\\
doubt that its recent performance levels, particularly in the Borders and Fife, have\\
been completely unacceptable. I have said that in the chamber and I heard Michael Matheson\\
say it a short while ago, when members including Annabelle Ewing raised legitimate and understandable\\
concerns on their constituents’ behalf.\\
We have used contractual mechanisms that are in the franchise agreement to require the\\
remedial plan. ScotRail will publish its performance remedial plan on its website shortly. The\\
commitments in that plan have been contracted as a remedial agreement. Of course, if ScotRail\\
does not achieve improved performance, or if it fails to deliver on its contractual\\
commitments, it runs the risk of its franchise being terminated early.\\
Rachael Hamilton (Ettrick, Roxburgh and Berwickshire) (Con):\\
I recently received a letter from a concerned Borders railway commuter. His letter says:\\
“It has come to the point where there is genuine surprise that the train is running\\
on time as opposed to it being so frequently cancelled ... The negative effects of this\\
are significant, there is a financial penalty imposed by the nursery as a result of collecting\\
my daughter”\\
late, and\\
“There is significant stress and anxiety because of the lateness at work”.\\
Will the First Minister apologise on behalf of her transport secretary to the hundreds\\
of commuters who are experiencing transport hell, and will she personally oversee the\\
remedial plan that has been submitted by ScotRail, and which will be published in the next few\\
days?\\
The First Minister: The transport secretary will oversee that,\\
because it is part of his responsibilities. However, as First Minister, I, too, will obviously\\
retain a very close interest in the matter.\\
I have made it very clear—I do not think that I can make it clearer—that some of\\
ScotRail’s recent performance levels have been completely unacceptable. That is particularly,\\
although not exclusively, the case on the Borders railway. I could stand here and talk\\
about some of the reasons for that, including train delivery and training requirements.\\
There have also been problems with trains coming into and going out of Edinburgh in\\
the past couple of days to do with Network Rail failings. However, I am not going to\\
talk about those reasons, because it is ScotRail’s responsibility to ensure that it lives up\\
to its performance standards. That is why the remedial plan is so important and why\\
ScotRail has to understand the seriousness of the obligation on it to deliver on the\\
commitments that it makes in the plan.\\
Mark Ruskell (Mid Scotland and Fife) (Green): Commuters are suffering from poor rail services\\
across Scotland, especially in Fife. Last year, the then Minister for Transport and\\
the Islands, Humza Yousaf, said in the chamber:\\
“there will be an upgrade in the rolling stock later in 2018 or early in 2019. Nevertheless,\\
people in Fife should not have to wait for that to get an improvement in their service”.—[Official\\
Report, 25 January 2018; c 2.]\\
Why are Fife commuters now being told that it will be the end of 2019 at the very earliest\\
before any improvements come through? Does not the First Minister believe that it is\\
time that she personally stepped in to take charge of the ScotRail crisis?\\
The First Minister: I have made my views clear, and I will do\\
so again. Those who are charged with and remunerated for the responsibility of running our railways\\
are the ones who have to get that right. They have a responsibility to do so and to begin\\
immediately to deliver the improvements that passengers want. That is what the remedial\\
plan will focus very much on.\\
Of course, significant investment is being made in our railways, with rolling stock being\\
renewed and a lot of other positive work being done from which I hope passengers will start\\
to benefit very soon. However, ScotRail must address the reasons for the dip in its performance—at\\
least, those that are within its responsibility—and we expect it to do so very quickly.\\
Liz Smith (Mid Scotland and Fife) (Con): To ask the First Minister, in light of the\\
parliamentary reports by both MSPs and MPs, what the Scottish Government's response is\\
to the growing concerns about the effects of social media on the mental health of young\\
people across Scotland. (S5F-03169)\\
The First Minister (Nicola Sturgeon): We welcome the report that was published last\\
week by the Public Audit and Post-legislative Scrutiny Committee on the relationship between\\
social media and mental health. It made recommendations on the need for further research in the area.\\
Next month, we will publish initial research on the links between unhealthy social media\\
use and lower mental wellbeing, in particular in girls and young women. We are also committed\\
to developing and publishing Scotland-specific advice on how young people can use social\\
media in a healthy way. That advice, which will be co-produced with young people for\\
young people, will be informed by the research that we will publish next month.\\
Liz Smith: I thank the First Minister for that helpful\\
answer. I am sure that the whole Parliament will be united in deep concern about the shocking\\
statistics that say that 60 per cent of 16 to 25-year-olds believe that social media\\
place “overwhelming pressure” on their age group, and that mental health referrals\\
have increased by 22 per cent since 2014. Those are just some of the facts that have,\\
quite rightly, led MSPs and MPs to state categorically that we all have a duty of care to protect\\
vulnerable users. In addition to her previous answer, can the First Minister give us some\\
details about the timescale that she envisages for implementing the task force delivery plan?\\
The First Minister: First, I appreciate very much the sentiments\\
behind, and the detail of, Liz Smith’s question. The internet and social media should be, and\\
in many respects are, forces for good that we should embrace and welcome, but they also\\
put considerable pressure on young people—in particular, young girls. Many of us have young\\
girls in our families—I have a niece who is about to enter her teenage years—and\\
it is not difficult to see that pressure. We must ensure that our young people are equipped\\
to deal with it properly.\\
I have referred to research that we will publish and work that will flow from it. The task\\
force is taking forward a substantial programme of work. I will ask the Minister for Mental\\
Health to write to Liz Smith with the precise timescales for delivery of the various aspects\\
of the work. All that work is important, so that we can prevent mental health issues and\\
provide treatment as quickly as possible, when it is required. Undoubtedly, part of\\
prevention is encouragement of, and support for, healthy use of social media.\\
Willie Coffey (Kilmarnock and Irvine Valley) (SNP):\\
The First Minister will be aware of the tragic death of a young 18-year-old girl in Kilmarnock\\
last weekend, which was followed only hours later by the death of another youngster in\\
Ayrshire. I understand that there are no suspicious circumstances. A growing number of young people\\
across Ayrshire seem to be ending their lives through suicide, which is clearly heartbreaking\\
for their families and friends. Will the First Minister offer some hope to youngsters and\\
their families by saying that services are there to help, and that if more can be done\\
to help to put an end to such awful tragedies, it will be done?\\
The First Minister: I give that assurance. I will not comment\\
on individual cases, beyond saying that my thoughts and condolences are with the families\\
involved. East Ayrshire Council is already looking at the incidents with the national\\
health service, and will want to ensure that it responds appropriately.\\
Across the chamber, we are committed to ensuring that, as the challenges around mental health\\
change and develop, our responses do so, too. I have said previously in the chamber that,\\
as the system has developed over many years, too many people are referred to specialist\\
services because there are not services in the community for prevention and early intervention.\\
Many of the initiatives that we are implementing through the investment that we announced recently\\
are trying to redress that balance, so that there is a focus on prevention and early intervention,\\
and so that we ensure that we also have specialist services when young people need them. I hope\\
and believe that that programme of work has wide support across the chamber.\\
The Presiding Officer: That concludes First Minister’s question\\
time. Before we move on to the members’ business debate, we will have a short suspension\\
to allow members, ministers and people in the gallery to change seats.}


\end{description}

\end{document}
